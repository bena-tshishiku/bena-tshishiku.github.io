\documentclass[11pt]{article}

\author{Math 1060}
\date{Due Friday, November 14 by 11:59pm} 
\title{Homework 7}

\usepackage{graphicx,xypic}
\usepackage{amsthm}
\usepackage{amsmath,amssymb}
\usepackage{amsfonts}
\usepackage{xcolor}
\usepackage[margin=1in]{geometry}
\usepackage[shortlabels]{enumitem}
\newtheorem{problem}{Problem}
\renewcommand*{\proofname}{{\color{blue}Solution}}


\setlength{\parindent}{0pt}
\setlength{\parskip}{1.25ex}


\begin{document}

\maketitle


% DON'T LEAVE THIS BLANK! 
{\bf\Large Your Name:} 

% DON'T FORGET YOUR COLLABORATORS! 
Collaborator names: 


\vspace{.3in}
Topics covered: Theorem Egregium, geodesics

Instructions {\bf (read these carefully!)}: 
\begin{itemize}
\item This assignment must be submitted on Gradescope by the due date. 
\item If you collaborate with other students (which is encouraged!), please list your collaborators above. 
\item Homework is graded anonymously, so please avoid putting your name on every page of the assignment.
\item If you are stuck, please ask for help (from me, Nathan, or a classmate). Use Campuswire!  
\item You may freely use any fact proved in class (mention it in the appropriate spot). You can also use basic facts from calculus or linear algebra without justification (but be sure to say what fact you're using and where). You may not freely use facts from the book, and generally your source material for completing the assignent should be the lecture, rather than the book (particularly when there is some discrepancy between these). 
\item Please restrict your solution to each problem to a single page. Usually solutions can be even shorter than that. If your solution is very long, you should think more about how to express it concisely.
\end{itemize}
\pagebreak 



\begin{problem}
Assume $f:S_1\to S_2$ is a differentiable bijection and is area-preserving\footnote{i.e.\ area of $f(R)$ equals area of $R$ for (reasonable) $R\subset S_1$}. Let $\phi:U\to S_1$ be a chart. Let $E_i,F_i,G_i$ be the corresponding first fundamental form of $S_i$ for $i=1,2$. Prove that $E_1G_1-F_1^2=E_2G_2-F_2^2$. 
\end{problem}

\begin{proof}

\end{proof}

\pagebreak

\begin{problem}
True or false: the Mobius band from the first homework can be made out of paper. Explain your answer. 
\end{problem}

\begin{proof}

\end{proof}

\pagebreak

\begin{problem}[dC, 4.4.3]
Show that the surfaces $\phi(u,v)=(u\cos v,u\sin v,\log u)$ and $\psi(u,v)=(u\cos v,u\sin v,v)$ have the same Gauss curvature, but $\psi\circ\phi^{-1}$ is not an isometry. How does this relate to the Theorem Egregium? \footnote{There is only one answer that is completely correct, so please think carefully.}
\end{problem}

\begin{proof}

\end{proof}

\pagebreak

\begin{problem}
Let $S$ be the torus of revolution. Determine which of the meridians and which of the longitudes on $S$ are geodesics. \footnote{If the revolution happens around the $z$-axis, longitudes are the circles on the torus with a fixed height (i.e.\ $z$-coordinate). Meridians are the circles that are being revolved.} 
\end{problem}

\begin{proof}

\end{proof}

\pagebreak

\begin{problem}
Show that the trefoil knot can be realized as a geodesic on a torus of revolution. Use \texttt{https://demonstrations.wolfram.com/GeodesicsOfATorusSolvedWithAMethodOfLagrange/}. Your solution should be a picture including all the parameters. Explain how you found your answer. \footnote{Hint: it may be helpful to think about how to find this systematically. } 
\end{problem}

\begin{proof}

\end{proof}




\end{document}