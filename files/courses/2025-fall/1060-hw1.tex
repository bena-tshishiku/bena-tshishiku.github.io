\documentclass[11pt]{article}

\author{Math 1060}
\date{Due Friday, September 12 by 11:59pm} 
\title{Homework 1}

\usepackage{graphicx,xypic}
\usepackage{amsthm}
\usepackage{amsmath,amssymb}
\usepackage{amsfonts}
\usepackage{xcolor}
\usepackage[margin=1in]{geometry}
\usepackage[shortlabels]{enumitem}
\newtheorem{problem}{Problem}
\renewcommand*{\proofname}{{\color{blue}Solution}}


\setlength{\parindent}{0pt}
\setlength{\parskip}{1.25ex}


\begin{document}

\maketitle


% DON'T LEAVE THIS BLANK! 
{\bf\Large Your Name:} 

% DON'T FORGET YOUR COLLABORATORS! 
Collaborator names: 


\vspace{.3in}
Topics covered: surface charts, tangent space

Instructions {\bf (read these carefully!)}: 
\begin{itemize}
\item This assignment must be submitted on Gradescope by the due date. Gradescope Entry Code: NG7JKV. 
\item If you collaborate with other students (which is encouraged!), please list your collaborators above. 
\item Homework is graded anonymously, so please avoid putting your name on every page of the assignment.
\item If you are stuck, please ask for help (from me, Nathan (TA), or a classmate). Use Campuswire!  
\item You may freely use any fact proved in class (mention it in the appropriate spot). You can also use basic facts from calculus or linear algebra without justification. You may not freely use facts from the book. 
\item Please restrict your solution to each problem to a single page. Usually solutions can be even shorter than that. If your solution is very long, you should think more about how to express it concisely.
\end{itemize}
\pagebreak 

\begin{problem}
Let $\alpha:I\to\mathbb R^3$ and $\beta:I\to\mathbb R^3$ be two curves. Let $\langle\alpha,\beta\rangle:I\to\mathbb R$ be the function defined by $\langle\alpha,\beta\rangle(t)=\langle\alpha(t),\beta(t)\rangle$, where $\langle\cdot,\cdot\rangle$ denotes the standard inner product on $\mathbb R^3$. Prove that 
\[\langle\alpha,\beta\rangle'(t)=\langle \alpha'(t),\beta(t)\rangle+\langle\alpha(t),\beta'(t)\rangle.\]
\end{problem}

\begin{proof}

\end{proof}

\pagebreak

\begin{problem}
Show that the cylinder $\{(x,y,z):x^2+y^2=1\}$ is a surface by covering it with two coordinate charts. 
\end{problem}

\begin{proof}

\end{proof}

\pagebreak

\begin{problem}
Let $N=(0,0,1)\in S^2$ be the north pole. Define stereographic projection $\pi: S^2\setminus\{N\}\to \mathbb R^2$ as follows. Given $p=(x,y,z)$, define $\pi(p)$ as the intersection of the line between $N$ and $p$ with the $xy$-plane.\footnote{Draw a picture.} Derive a formula for the inverse of $\pi$, and check that it is a chart.  \footnote{This problem gives yet another way to cover the sphere by coordinate charts.} 
\end{problem}

\begin{proof}

\end{proof}

\pagebreak

\begin{problem}
Show that the tangent plane of the graph of a function $f:\mathbb R^2\to\mathbb R$ at $p=(u,v,f(u,v))$ is the graph of the differential $Df_{(u,v)}$. 
\end{problem}

\begin{proof}

\end{proof}

\pagebreak

\begin{problem}
Derive a formula for a differentiable map from a rectangle in $\mathbb R^2$ whose image is a M\"obius strip in $\mathbb R^3$. Do the same for an annulus with two twists. Include an image of your surfaces, plotted using \texttt{ParametricPlot3D} in Mathematica, or similar. 
\end{problem}


\begin{proof}

\end{proof}


\end{document}