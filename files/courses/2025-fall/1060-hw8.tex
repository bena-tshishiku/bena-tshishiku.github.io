\documentclass[11pt]{article}

\author{Math 1060}
\date{Due Friday, November 21 by 11:59pm} 
\title{Homework 8}

\usepackage{graphicx,xypic}
\usepackage{amsthm}
\usepackage{amsmath,amssymb}
\usepackage{amsfonts}
\usepackage{xcolor}
\usepackage[margin=1in]{geometry}
\usepackage[shortlabels]{enumitem}
\newtheorem{problem}{Problem}
\renewcommand*{\proofname}{{\color{blue}Solution}}


\setlength{\parindent}{0pt}
\setlength{\parskip}{1.25ex}


\begin{document}

\maketitle


% DON'T LEAVE THIS BLANK! 
{\bf\Large Your Name:} 

% DON'T FORGET YOUR COLLABORATORS! 
Collaborator names: 


\vspace{.3in}
Topics covered: Theorem Egregium, geodesics, parallel transport

Instructions {\bf (read these carefully!)}: 
\begin{itemize}
\item This assignment must be submitted on Gradescope by the due date. 
\item If you collaborate with other students (which is encouraged!), please list your collaborators above. 
\item Homework is graded anonymously, so please avoid putting your name on every page of the assignment.
\item If you are stuck, please ask for help (from me, Nathan, or a classmate). Use Campuswire!  
\item You may freely use any fact proved in class (mention it in the appropriate spot). You can also use basic facts from calculus or linear algebra without justification (but be sure to say what fact you're using and where). You may not freely use facts from the book, and generally your source material for completing the assignent should be the lecture, rather than the book (particularly when there is some discrepancy between these). 
\item Please restrict your solution to each problem to a single page. Usually solutions can be even shorter than that. If your solution is very long, you should think more about how to express it concisely.
\end{itemize}
\pagebreak 



\begin{problem}
Let $v,w$ be vector fields along a curve $c:I\to S$. Prove\footnote{Hint: sometimes it's good to work in coordinates, and sometimes not..} that 
\[\frac{d}{dt}\langle v(t), w(t)\rangle=\langle \nabla_\alpha v, w\rangle+\langle v,\nabla_\alpha w\rangle.\]
\end{problem}

\begin{proof}

\end{proof}

\pagebreak

\begin{problem}
Let $S$ be the cylinder $x^2+y^2=1$ and let $C$ be the curve obtained by intersecting $S$ with the plane $x-z=0$. Compute the geodesic curvature of $C$ at the point $(1,0,1)$.  
\end{problem}

\begin{proof}

\end{proof}

\pagebreak

\begin{problem}
Let $S$ be a surface, and suppose $\phi:U\to S$ is a coordinate chart whose first fundamental form satisfies $F=0$ and $E=\lambda=G$ for some function $\lambda$.\footnote{This is called an isothermal chart.} 
\begin{enumerate}[(a)]
\item Prove that $\phi_{uu}+\phi_{vv}$ is orthogonal to $\phi_u$ and $\phi_v$. \footnote{Write the assumption as a function you can d.i.f.f.e.r.e.n.t.i.a.t.e.} 
\item By (a), $\phi_{uu}+\phi_{vv}=\mu N$ for some $\mu$. Compute $\mu$. 
\item Show that if $S$ is a minimal surface, then $\phi$ is harmonic, i.e.\ $\phi_{uu}+\phi_{vv}=0$. 
\end{enumerate} 
\end{problem}

\begin{proof}

\end{proof}

\pagebreak

\begin{problem}
Let $\phi:U\to S$ be an isothermal chart. Prove that 
\[K=\frac{-1}{2\lambda}\Delta(\log\lambda),\]
where $\Delta f=f_{uu}+f_{vv}$ is the Laplacian. \footnote{Hint: Use the formula for $K$ in terms of $E$ and the Christoffel symbols. Then write the Christoffel symbols in terms of $E,F,G$ and their derivatives.}
\end{problem}

\begin{proof}

\end{proof}


\pagebreak

\begin{problem}
Define third fundamental form on $T_pS$ by $\mathrm{I\!I\!I}_p(x,y)= \langle DN_p(x), DN_p(y)\rangle$. Prove with an explicit formula\footnote{The formula should be exceedingly pleasant} that the third fundamental form can be expressed in terms of the first and second fundamental forms. \footnote{Hint: Use the Cayley--Hamilton theorem from linear algebra.}
\end{problem}

\begin{proof}

\end{proof}

\pagebreak

\begin{problem}
Let $S$ be the hyperboloid $x^2+y^2=z^2+1$. Find a geodesic on $S$ that is a straight line. Prove that $S$ is a union of straight lines. Use this to give a method for boiling pasta that prevents the noodles from sticking to each other. 
\end{problem}

\begin{proof}

\end{proof}






\end{document}