\documentclass[11pt]{article}

\author{Math 1060}
\date{Due Friday, October 17 by 11:59pm} 
\title{Homework 5}

\usepackage{graphicx,xypic}
\usepackage{amsthm}
\usepackage{amsmath,amssymb}
\usepackage{amsfonts}
\usepackage{xcolor}
\usepackage[margin=1in]{geometry}
\usepackage[shortlabels]{enumitem}
\newtheorem{problem}{Problem}
\renewcommand*{\proofname}{{\color{blue}Solution}}


\setlength{\parindent}{0pt}
\setlength{\parskip}{1.25ex}


\begin{document}

\maketitle


% DON'T LEAVE THIS BLANK! 
{\bf\Large Your Name:} 

% DON'T FORGET YOUR COLLABORATORS! 
Collaborator names: 


\vspace{.3in}
Topics covered: second fundamental form, normal curvature, Gaussian curvature

Instructions {\bf (read these carefully!)}: 
\begin{itemize}
\item This assignment must be submitted on Gradescope by the due date. 
\item If you collaborate with other students (which is encouraged!), please list your collaborators above. 
\item Homework is graded anonymously, so please avoid putting your name on every page of the assignment.
\item If you are stuck, please ask for help (from me, Nathan, or a classmate). Use Campuswire!  
\item You may freely use any fact proved in class (mention it in the appropriate spot). You can also use basic facts from calculus or linear algebra without justification (but be sure to say what fact you're using and where). You may not freely use facts from the book, and generally your source material for completing the assignent should be the lecture, rather than the book (particularly when there is some discrepancy between these). 
\item Please restrict your solution to each problem to a single page. Usually solutions can be even shorter than that. If your solution is very long, you should think more about how to express it concisely.
\end{itemize}
\pagebreak 





\begin{problem}
For the monkey saddle $z=x^3-3xy^2$, compute the plane-curve curvature of the curve on $S$ lying over each line through the origin in the $xy$-plane.\footnote{We did a similar computation in class for the hyperboloid.} Conclude that the monkey saddle has Gaussian curvature $K=0$ at the origin. 
\end{problem}

\begin{proof}

\end{proof}

\pagebreak

\begin{problem}
Let $S$ be a surface, and fix $q\in\mathbb R^3$. Define $f:S\to\mathbb R$ by $f(p)=|p-q|^2$. Compute $Df_p$.\footnote{Here (in our usual notation) $Df_p:T_pS\to\mathbb R$ is defined on $w=c'(0)$ by $Df_p(w)=(f\circ c)'(0)$.} Describe geometrically when is $p$ a critical point\footnote{We say that $p\in S$ is a critical point of $f:S\to\mathbb R$ if $Df_p=0$.}  of $f$.
\end{problem}

\begin{proof}

\end{proof}

\pagebreak

\begin{problem}
Use the previous problem to show that a closed, bounded surface $S\subset\mathbb R^3$ has a point with $K(p)\ge0$. \footnote{We sketched a proof in class. Here I want you to give a rigorous argument.} 
\end{problem}

\begin{proof}

\end{proof}

\pagebreak


\begin{problem}
Let $S\subset\mathbb R^3$ be a surface. Suppose that there exists a point $q\in\mathbb R^3$ such that the normal line through $p\in S$ passes through $q$ for each $p\in S$. Prove that $S$ is contained in a sphere. \footnote{Hint: Define an appropriate function on $S$ whose derivative can be computed as zero...}
\end{problem}

\begin{proof}

\end{proof}

\pagebreak

\begin{problem}
True or false: For a surface $S\subset\mathbb R^3$ and $p\in S$, if the Gauss curvature $K(p)$ is negative, then there exists a curve through $p$ whose normal curvature at $p$ is zero. 
\end{problem}

\begin{proof}

\end{proof}





\end{document}