\documentclass[11pt]{article}

\author{Math 106}
\date{Due Friday, Nov 3 by 11:59pm} 
\title{Homework 7}

\usepackage{graphicx,xypic}
\usepackage{amsthm}
\usepackage{amsmath,amssymb}
\usepackage{amsfonts}
\usepackage{xcolor}
\usepackage[margin=1in]{geometry}
\usepackage[shortlabels]{enumitem}
\newtheorem{problem}{Problem}
\renewcommand*{\proofname}{{\color{blue}Solution}}



\DeclareMathOperator{\sech}{sech}


\setlength{\parindent}{0pt}
\setlength{\parskip}{1.25ex}


\begin{document}

\maketitle

% You are required to put your name here:
{\bf\Large Name:} 


\vspace{.3in}
Topics covered: curvature, metric properties of charts

Instructions: 
\begin{itemize}
\item This assignment must be submitted on Gradescope by the due date. 
\item If you collaborate with other students (which is encouraged!), please mention this near the corresponding problems. 
\item If you are stuck, please ask for help (from me, a TA, a classmate). Use ed discussions!  
\item You may freely use any fact proved in class. In general, you should provide proof for facts used that were not proved in class. 
\item Please restrict your solution to each problem to a single page. Usually solutions can be even shorter than that. If your solution is very long, you should think more about how to express it concisely.
\end{itemize}
\pagebreak 


\begin{problem}
True or false: For a surface $S\subset\mathbb R^3$ and $p\in S$, if the Gauss curvature $K(p)$ is negative, then there exists a nonzero vector $w\in T_pS$ whose normal curvature is zero. 
\end{problem}

\begin{proof}

\end{proof}

\pagebreak 

\begin{problem}[dC, 3.5.12]
Prove that every minimal surfaces $S\subset\mathbb R^3$ is unbounded. \footnote{Hint: We proved a helpful fact about curvature of closed, bounded surfaces in class.}
\end{problem}

\begin{proof}

\end{proof}

\pagebreak 


\begin{problem}[dC, 4.2.4]
Show that the stereographic projection chart $\mathbb R^2\to S^2$ is conformal. 
\end{problem}

\begin{proof}

\end{proof}

\pagebreak 

\begin{problem}[dC, 4.2.19]
Consider the cylinder $C=\{x^2+y^2=1\}$, and let $M$ be $S^2$ without the north and south poles. Given $p=(x,y,z)\in M$, let $R_p$ be the ray based at $(0,0,z)$ and going through $p$. Define $f:M\to C$ by $f(p)=R_p\cap C$. Prove that $f$ is an area preserving map. Use this to give a quick computation of the area of $S^2$. 
\end{problem}

\begin{proof}

\end{proof}

\pagebreak 

\begin{problem}
Let $S_1,S_2$ be surfaces. Let $\phi:U\to S_1$ be a chart, and let $f:S_1\to S_2$ be a smooth bijection. Let $E_1,F_1,G_1$ and $E_2,F_2,G_2$ be the first fundamental forms of $S_1$ and $S_2$ with respect to the charts $\phi$ and $f\circ\phi$, respectively. Prove that if $f$ is area preserving, then $E_1G_1-F_1^2=E_2G_2-F_2^2$. 
\end{problem}

\begin{proof}

\end{proof}




\end{document}