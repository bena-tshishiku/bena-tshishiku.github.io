\documentclass[11pt]{article}

\author{Math 106}
\date{Due Friday, Oct 19 by 11:59pm} 
\title{Homework 6}

\usepackage{graphicx,xypic}
\usepackage{amsthm}
\usepackage{amsmath,amssymb}
\usepackage{amsfonts}
\usepackage{xcolor}
\usepackage[margin=1in]{geometry}
\usepackage[shortlabels]{enumitem}
\newtheorem{problem}{Problem}
\renewcommand*{\proofname}{{\color{blue}Solution}}



\DeclareMathOperator{\sech}{sech}


\setlength{\parindent}{0pt}
\setlength{\parskip}{1.25ex}


\begin{document}

\maketitle

% You are required to put your name here:
{\bf\Large Name:} 


\vspace{.3in}
Topics covered: second fundamental form, surface curvatures

Instructions: 
\begin{itemize}
\item This assignment must be submitted on Gradescope by the due date. 
\item If you collaborate with other students (which is encouraged!), please mention this near the corresponding problems. 
\item If you are stuck, please ask for help (from me, a TA, a classmate). Use ed discussions!  
\item You may freely use any fact proved in class. In general, you should provide proof for facts used that were not proved in class. 
\item Please restrict your solution to each problem to a single page. Usually solutions can be even shorter than that. If your solution is very long, you should think more about how to express it concisely.
\end{itemize}
\pagebreak 


\begin{problem}
The helicoid is the surface given by the chart
\[\phi(u,v)=(v\cos u,v\sin u,u), \>\>\>u,v\in\mathbb R.\]
Use a mathematica \texttt{ParametricPlot3D} (or similar) to plot this surface. Compute (by hand) the first and second fundamental forms $\mathrm I,\mathrm{I\!I}$ and mean curvature $H$ of this surface.\footnote{Hint: Your answer for the mean curvature, if correct, will be exceedingly simple.} 
\end{problem}

\begin{proof}

\end{proof}

\pagebreak


\begin{problem}
Consider the curve\footnote{Recall the hyperbolic trig functions are defined by $\cosh(t)=\frac{e^t+e^{-t}}{2}$, $\sinh(t)=\frac{e^t-e^{-t}}{2}$, etc. I suggest you look up formulas for the derivatives and identities satisfied by these functions.}
\[\alpha(t)=(t-\tanh t,\sech t,0),\>\>\>t>0.\]
Let $S$ be the surface obtained by revolving $\alpha$ about the $x$-axis. Use a mathematica \texttt{ParametricPlot3D} (or similar) to plot this surface. Compute (by hand) the first and second fundamental forms $\mathrm I,\mathrm{I\!I}$ and Gauss curvature $K$ of this surface.\footnote{Hint: Your answer for the Gauss curvature, if correct, will be exceedingly simple.} 
\end{problem}

\begin{proof}

\end{proof}

\pagebreak

\begin{problem}
Let $S$ be a surface with a unit normal $N:S\to S^2$. Let $\alpha:I\to S$ be a curve. 
Assume that $\alpha'(t)$ is a principal direction for each $t$.\footnote{Remark: In this case, $\alpha$ is a called a \emph{line of curvature}.}  Show that the curvature $\kappa=\kappa_\alpha$ of $\alpha$ satisfies $\kappa=|k_nk_N|$, where $k_n$ is the normal curvature and $k_N$ is the curvature of $N\circ\alpha$. \footnote{Hint: note that $\alpha$ and $N\circ\alpha$ are not necessarily unit speed. Use a formula for curvature from a previous problem.} 
\end{problem}

\begin{proof}

\end{proof}

\pagebreak

\begin{problem}
Let $S$ be a surface, and fix $q\in\mathbb R^3$. Define $f:S\to\mathbb R$ by $f(p)=|p-q|^2$. Give a formula for $Df_p(w)$ directly using the way we defined the derivative of a function on a surface in class. When is $p$ a critical point\footnote{We say that $p\in S$ is a critical point of $f:S\to\mathbb R$ if $Df_p=0$.}  of $f$? 
\end{problem}

\begin{proof}

\end{proof}

\pagebreak

\begin{problem}
Let $S\subset\mathbb R^3$ be a surface. Suppose that there exists a point $q\in\mathbb R^3$ such that the normal line through $p\in S$ passes through $q$ for each $p\in S$. Prove that $S$ is contained in a sphere. \footnote{Hint: Define an appropriate function on $S$ whose derivative is constant...}
\end{problem}

\begin{proof}

\end{proof}



\end{document}