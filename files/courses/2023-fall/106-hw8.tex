\documentclass[11pt]{article}

\author{Math 106}
\date{Due Friday, Nov 10 by 11:59pm} 
\title{Homework 8}

\usepackage{graphicx,xypic}
\usepackage{amsthm}
\usepackage{amsmath,amssymb}
\usepackage{amsfonts}
\usepackage{xcolor}
\usepackage[margin=1in]{geometry}
\usepackage[shortlabels]{enumitem}
\newtheorem{problem}{Problem}
\renewcommand*{\proofname}{{\color{blue}Solution}}



\DeclareMathOperator{\sech}{sech}


\setlength{\parindent}{0pt}
\setlength{\parskip}{1.25ex}


\begin{document}

\maketitle

% You are required to put your name here:
{\bf\Large Name:} 


\vspace{.3in}
Topics covered: chart geometry, Theorem Egregium, final projects

Instructions: 
\begin{itemize}
\item This assignment must be submitted on Gradescope by the due date. 
\item If you collaborate with other students (which is encouraged!), please mention this near the corresponding problems. 
\item If you are stuck, please ask for help (from me, a TA, a classmate). Use ed discussions!  
\item You may freely use any fact proved in class. In general, you should provide proof for facts used that were not proved in class. 
\item Please restrict your solution to each problem to a single page. Usually solutions can be even shorter than that. If your solution is very long, you should think more about how to express it concisely.
\end{itemize}
\pagebreak 

\begin{problem}
Let $w$ be a vector field on a surface $S$. Given a smooth function $f:S\to\mathbb R$, define $w(f):S\to\mathbb R$ by 
\[w(f)(p)=(f\circ\alpha)'(0)\]
where $\alpha:I\to S$ is a curve such that $\alpha(0)=p$ and $\alpha'(0)=w(p)$. For functions $f,g$ and real numbers $\lambda,\mu$, prove 
\[w(\lambda f+\mu g)=\lambda w(f)+\mu w(g)\>\>\>\text{ and }\>\>\>w(fg)=w(f)g+fw(g).\]
Explain the significance of (a) from a linear algebra point-of-view.
\end{problem}

\begin{proof}

\end{proof}

\pagebreak

\begin{problem}
True or false: the Mobius band from HW5 can be made out of a strip of paper by gluing the ends. Explain your answer. 
\end{problem}

\begin{proof}

\end{proof}

\pagebreak


\begin{problem}
Let $\phi:U\to S^2$ be a spherical coordinates chart. Compute in coordinates the functions $\phi_{uu},\phi_{uv},\phi_{vv},N_u,N_v$ as linear combinations of $\phi_u,\phi_v,N$. 
\end{problem}

\begin{proof}

\end{proof}

\pagebreak



\begin{problem}[dC, 4.4.3]
Show that the surfaces $\phi(u,v)=(u\cos v,u\sin v,\log u)$ and $\psi(u,v)=(u\cos v,u\sin v,v)$ have the same Gauss curvature, but $\psi\circ\phi^{-1}$ is not an isometry. \footnote{This shows the converse of the Theorem Egregium is false.} 
\end{problem}

\begin{proof}

\end{proof}

\pagebreak

\begin{problem}
Let $S$ be a surface, and suppose $\phi:U\to S$ is a coordinate chart whose first fundamental form satisfies $F=0$ and $E=\lambda=G$ for some function $\lambda$.\footnote{This is called an isothermal chart. Such a chart always exists, but this is not so easy to prove.} 
\begin{enumerate}[(a)]
\item Prove that $\phi_{uu}+\phi_{vv}$ is orthogonal to $\phi_u$ and $\phi_v$. \footnote{Hint: use the partial derivatives of the functions $\phi_u\cdot\phi_v$ and $\phi_u\cdot\phi_u=\phi_v\cdot\phi_v$.} 
\item By (a), $\phi_{uu}+\phi_{vv}=\mu N$ for some $\mu$. Compute $\mu$. 
\item Show that if $S$ is a minimal surface, then $\phi$ is harmonic, i.e.\ $\phi_{uu}+\phi_{vv}=0$. \footnote{Technically, it may be better to say the coordinate functions of $\phi$ are harmonic.} 
\end{enumerate} 
\end{problem}

\begin{proof}

\end{proof}

\pagebreak

{\bf Submit a final project outline by Monday 11/13. See other document (course webpage) for further instructions. }



\end{document}