\documentclass[11pt]{article}

\author{Math 106}
\date{Due Friday, Nov 17 by 11:59pm} 
\title{Homework 9}

\usepackage{graphicx,xypic}
\usepackage{amsthm}
\usepackage{amsmath,amssymb}
\usepackage{amsfonts}
\usepackage{xcolor}
\usepackage[margin=1in]{geometry}
\usepackage[shortlabels]{enumitem}
\newtheorem{problem}{Problem}
\renewcommand*{\proofname}{{\color{blue}Solution}}



\DeclareMathOperator{\sech}{sech}


\setlength{\parindent}{0pt}
\setlength{\parskip}{1.25ex}


\begin{document}

\maketitle

% You are required to put your name here:
{\bf\Large Name:} 


\vspace{.3in}
Topics covered: curvature, geodesics

Instructions: 
\begin{itemize}
\item This assignment must be submitted on Gradescope by the due date. 
\item If you collaborate with other students (which is encouraged!), please mention this near the corresponding problems. 
\item If you are stuck, please ask for help (from me, a TA, a classmate). Use ed discussions!  
\item You may freely use any fact proved in class. In general, you should provide proof for facts used that were not proved in class. 
\item Please restrict your solution to each problem to a single page. Usually solutions can be even shorter than that. If your solution is very long, you should think more about how to express it concisely.
\end{itemize}
\pagebreak 

\begin{problem}
Let $S$ be the torus of revolution. Determine which of the longitudes on $S$ are geodesics.\footnote{The longitudes are the circles on the torus with constant $z$-coordinate. } Be sure to explain your answer. 
\end{problem}

\begin{proof}

\end{proof}




\begin{problem}
Let $v,w$ be vector fields along a curve $\alpha:I\to S$. Prove\footnote{For the love of algebra, don't do the computation in coordinates.} that 
\[\frac{d}{dt}\big[v(t)\cdot w(t)\big]=\nabla_\alpha v\cdot w+v\cdot\nabla_\alpha w.\]
\end{problem}

\begin{proof}

\end{proof}



\begin{problem}
Define third fundamental form on $T_pS$ by $\mathrm{I\!I\!I}_p(x,y)= DN_p(x)\cdot DN_p(y)$. Prove with an explicit formula\footnote{The formula should be exceedingly pleasant} that the third fundamental form can be expressed in terms of the first fundamental form. \footnote{Hint: Use the Cayley--Hamilton theorem from linear algebra. If you don't know what this says, you should look it up.}
\end{problem}

\begin{proof}

\end{proof}

\begin{problem}
Let $\phi$ be a parameterization with $F=0$ and $E=\lambda=G$ for some function $\lambda$. Prove that 
\[K=\frac{-1}{2\lambda}\Delta(\log\lambda),\]
where $\Delta f=f_{uu}+f_{vv}$ is the Laplacian. \footnote{Hint: Use the formula for $K$ in terms of $E$ and the Christoffel symbols. Then write the Christoffel symbols in terms of $E,F,G$ and their derivatives.}
\end{problem}

\begin{proof}

\end{proof}


\begin{problem}
Let $S$ be the hyperboloid $x^2+y^2=z^2+1$. Find a geodesic on $S$ that is a straight line. Prove that $S$ is a union of straight lines. Use this to give a method for boiling pasta that prevents the noodles from sticking to each other. 
\end{problem}

\begin{proof}

\end{proof}

\pagebreak

{\bf Submit a draft of your final project slides. See other document (course webpage) for further instructions. }


\end{document}