\documentclass[11pt]{article}

\author{Math 123}
\date{Due February 10, 2023 by 5pm} 
\title{Homework 2}

\usepackage{graphicx,xypic}
\usepackage{amsthm}
\usepackage{amsmath,amssymb}
\usepackage{amsfonts}
\usepackage{xcolor}
\usepackage[margin=1in]{geometry}
\usepackage[shortlabels]{enumitem}
\newtheorem{problem}{Problem}
\renewcommand*{\proofname}{{\color{blue}Solution}}


\usepackage{fancyhdr}
\pagestyle{fancy}
\rhead{Math 123, Homework 2}

\setlength{\parindent}{0pt}
\setlength{\parskip}{1.25ex}


\begin{document}

\maketitle

% You are required to put your name here:
{\bf\Large Name:} 


\vspace{.3in}
Topics covered: bipartite graphs, Euler tours, degree sum formula, trees

Instructions: 
\begin{itemize}
\item This assignment must be submitted on Gradescope by the due date. 
\item If you collaborate with other students (which is encouraged!), please mention this near the corresponding problems. You must type your solutions alone. 
\item If you are stuck, please ask for help (from me, a TA, a classmate). Use Campuswire!  
\end{itemize}
\pagebreak 



\begin{problem}
The complete bipartite graph $K_{n,m}$ is the graph with $n+m$ vertices $v_1,\ldots,v_n$ and $u_1,\ldots,u_m$ and edges $\{v_i,u_j\}$ for each $1\le i\le n$ and $1\le j\le m$. Determine the values $n,m$ so that $K_{n,m}$ is Eulerian.
\end{problem}

\begin{proof}

\end{proof} 

\begin{problem}
Prove or disprove: 
\begin{enumerate}[(a)]
\item Every Eulerian bipartite graph has an even number of edges. 
\item Every Eulerian graph with an even number of vertices has an even number of edges. 
\end{enumerate} 
\end{problem}

\begin{proof}


\end{proof}


\begin{problem}Prove that every tree has at least two vertices of degree $1$. Classify trees with exactly two vertices of degree $1$. \footnote{Useful fact: a tree with $n$ vertices has $n-1$ edges. We will show this in class next week.}
\end{problem}

\begin{proof}

\end{proof}

\begin{problem}
Determine the number of graphs with $7$-vertices, each of degree $4$ (up to isomorphism). \footnote{Hint: consider the complement. Your solution should not be long. Use may want to use the previous problem.} 
\end{problem}

\begin{proof}

\end{proof}

\begin{problem}
Use induction on the number of edges to prove that a graph with no odd cycle is bipartite. 
\end{problem}


\begin{proof}

\end{proof}

\begin{problem}
Suppose there are two mountain trails, each starting at sea level and ending at the same elevation. Suppose hikers $A$,$B$ start hiking these two different trails at the same time. The Mountain Climber Problem asks if it is possible for $A$ and $B$ to hike to the top of their individual trails in a way so that they have the same elevation at every time.\footnote{It is important to note that the hikers are allowed to backtrack.} We model the trails by functions $f,g:[0,1]\rightarrow[0,1]$ with $f(0)=g(0)=0$ and $f(1)=g(1)=1$. In this problem you solve the Mountain Climber Problem in the case when $f$ and $g$ are piecewise linear continuous functions.\footnote{A function $f:[0,1]\rightarrow\mathbb R$ is piecewise linear if it's possible to express $[0,1]$ as a union of finitely many intervals, so that $f$ is linear ($x\mapsto ax+b$) on each. }
\begin{enumerate}[(a)]
\item Consider \[Z=\{(x,y)\in[0,1]\times[0,1]: f(x)=g(y)\}\]
Assuming $f,g$ are piecewise linear, determine the local picture near $(x,y)$ in $Z$, considering cases based on the local pictures of $f$ and $g$ near $x$ and $y$, respectively.
\item Observe that $Z$ can be given the structure of a graph $G$. Show that $G$ has exactly two vertices of odd degree. Deduce that there is a path in $G$ from $(0,0)$ to $(1,1)$.
\end{enumerate} 
\end{problem}

\begin{proof}

\end{proof}



\end{document}