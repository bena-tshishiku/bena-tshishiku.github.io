\documentclass[11pt]{article}

\author{Math 123}
\date{Due April 21, 2023 by midnight} 
\title{Homework 10}

\usepackage{graphicx,xypic}
\usepackage{amsthm}
\usepackage{amsmath,amssymb}
\usepackage{amsfonts}
\usepackage{xcolor}
\usepackage[margin=1in]{geometry}
\usepackage[shortlabels]{enumitem}
\newtheorem{problem}{Problem}
\renewcommand*{\proofname}{{\color{blue}Solution}}


\usepackage{fancyhdr}
\pagestyle{fancy}
\rhead{Math 123, Homework 10}

\setlength{\parindent}{0pt}
\setlength{\parskip}{1.25ex}


\begin{document}

\maketitle

% You are required to put your name here:
{\bf\Large Name:} 


\vspace{.3in}
Topics covered: Ramsey theory, random graphs

Instructions: 
\begin{itemize}
\item This assignment must be submitted on Gradescope by the due date. 
\item If you collaborate with other students (which is encouraged!), please mention this near the corresponding problems. 
\item Some problems from this assignment come from West's book, as indicated next to the problem. In some cases, the statements on this assignment differ slightly from the book. 
\item If you are stuck please ask for help (from me or your classmates). Occasionally problems may require ingredients not discussed in the course. 
\item You may freely use any fact proved in class. In general, you should provide proof for facts used that were not proved in class. 
\end{itemize}

\pagebreak 

\begin{problem}
Prove $R(4,4)>17$ using the $17$-vertex graph described in class.\footnote{17 vertices around a circle; connected a given vertex to the vertices distance 1, 2, 4, 8 away.}
\end{problem}

\begin{proof}

\end{proof}

\pagebreak

\begin{problem}
Fix a graph $H$ with $k$ vertices. Prove that almost every graph contains $H$ as an induced subgraph.\footnote{Recall: given a collection of vertices in a graph $G$, the induced subgraph is the subgraph consisting of those vertices and all the edges between them that belong to $G$.} \footnote{Hint: Decompose the vertices into groups of size $k$. Consider the event that one these groups spans $H$.}
\end{problem}

\begin{proof}

\end{proof}



\pagebreak

\begin{problem}
Recall that a graph $G$ satisfies property $(\star)$ if for any collection $u_1,\ldots,u_p$ and $v_1,\ldots,v_q$ of distinct vertices of $G$ there exists a vertex $z$ of $G$ so that $z$ is adjacent to all of the $u_i$ and to none of the $v_j$. Let $G_1,G_2$ be graphs whose vertex sets are countably infinite. Prove that if $G_1$ and $G_2$ satisfy $(\star)$, then $G_1$ and $G_2$ are isomorphic.\footnote{Hint: Enumerate the vertices of $G_1$ and $G_2$ by $x_1,x_2,\ldots$ and $y_1,y_2,\ldots$, respectively. Inductively define an isomorphism $f:V(G_1)\rightarrow V(G_2)$. On odd (resp.\ even) steps of the induction extend $f$ so that the smallest unmatched vertex of $V(G_1)$ (resp.\ $V(G_2)$) is in the domain (resp.\ image) of $f$.}
\end{problem}

\begin{proof}

\end{proof}

\pagebreak

\begin{problem}
Prove that the Rado graph has the following ``pigeonhole" property: For any partition of the vertex set $V=U_1\cup\cdots\cup U_m$, there exists $j$ so that the subgraph spanned by $U_j$ is isomorphic to $R$. 
\end{problem}

\begin{proof}

\end{proof}

\pagebreak


\begin{problem}[Bonus] Create a meme related to the course. Please submit to the campuswire page for everyone's enjoyment. 
\end{problem}

\pagebreak
{\it Submit a draft of your final project slides. See other document for instructions.} 


\end{document}