\documentclass[11pt]{article}

\author{Math 1030}
\date{Due Friday, January 30 by 5pm} 
\title{Homework 1}

\usepackage{graphicx,xypic}
\usepackage{amsthm}
\usepackage{amsmath,amssymb}
\usepackage{amsfonts}
\usepackage{xcolor}
\usepackage[margin=1in]{geometry}
\usepackage[shortlabels]{enumitem}
\newtheorem{problem}{Problem}
\renewcommand*{\proofname}{{\color{blue}Solution}}


\setlength{\parindent}{0pt}
\setlength{\parskip}{1.25ex}


\begin{document}

\maketitle

% DON'T LEAVE THIS BLANK! 
{\bf\Large Your Name:} 

% DON'T FORGET YOUR COLLABORATORS! 
Collaborator names: 

\vspace{.3in}
Topics covered: graph/subgraph, cycle, path, vertex degrees, isomorphism

Instructions {\bf (read these carefully!)}: 
\begin{itemize}
\item This assignment must be submitted on Gradescope by the due date. Gradescope Entry Code: 6X8XYR. 
\item If you collaborate with other students (this is encouraged),  list your collaborators above. 
\item Homework is graded anonymously, so avoid putting your name on each page of the assignment.
\item If you are stuck, please ask for help (from me, Rafi, or a classmate). Use Ed discussions!  
\item You may freely use any fact proved in class (mention it in the appropriate spot). You \emph{may not} freely use facts from the book or other sources.
\item As a general rule, try to restrict your solution to each problem to a single page. Often solutions can be short, while still being complete. If your solution is very long, think more about how you might express it more concisely.
\end{itemize}

\pagebreak 




\begin{problem}
Prove that the following graph is isomorphic to the Petersen graph.
\begin{center}
\includegraphics[scale=.4]{petersen.pdf}
\end{center}
\end{problem}

\begin{proof}

\end{proof}

\pagebreak 


\begin{problem}
How many cycles of length $n$ are there in the complete graph $K_n$? (Explain your answer.) 
\end{problem}

\begin{proof}

\end{proof}

\pagebreak 


\begin{problem}
Define the hypercube graph $Q_k$ as the graph with a vertex for each tuple $(a_1,\ldots,a_k)$ with coordinates $a_i\in\{0,1\}$ and with an edge between $(a_1,\ldots,a_k)$ and $(b_1,\ldots,b_k)$ if they differ in exactly one coordinate.
\begin{enumerate}[(a)]
\item Prove that two $4$-cycles in $Q_k$ are either disjoint, intersect in a single vertex, or intersect in a single edge.
\item Let $K_{2,3}$ be the complete bipartite graph with $2$ red and $3$ blue vertices. Prove that $K_{2,3}$ is not a subgraph of any hypercube $Q_k$. 
\end{enumerate} 
\end{problem}

\begin{proof}

\end{proof}

\pagebreak 


\begin{problem}
Let $G=(V,E)$ be a graph. The complement of $G$ is the graph $\bar G=(V,\bar E)$, where $\{u,v\}\in\bar E$ if and only if $\{u,v\}\notin E$. 
\begin{enumerate}[(a)]
\item Determine the complement of the graphs $P_3$ and $P_4$. (Recall that $P_n$ is the path with $n$ vertices.)
\item We say that $G$ is self-complementary if $G$ is isomorphic $\bar G$. Prove that if $G$ is self-complementary with $n$ vertices, then either $n$ or $n-1$ is divisible by $4$. 
\item Construct a self-complementary graph for each $n$ such that $n$ or $n-1$ is divisible by $4$.\footnote{Hint: The first interesting case is $n=5$. Build an example using $P_4$ and one more vertex. Study this case carefully and generalize. For $n=8$ you can start with two copies of $P_4$.}
\end{enumerate} 
\end{problem}

\begin{proof}

\end{proof}

\pagebreak 


\begin{problem}True or false: if $G$ is isomorphic to $H$, then the complements $\bar G$ and $\bar H$ are also isomorphic.\footnote{True/false questions require either a proof (if the statement is true) or a counterexample (if the statement is false). }
\end{problem}

\begin{proof}

\end{proof}

\pagebreak 


\begin{problem}
For this problem, let $G$ denote the Petersen graph. Use the definition of the Petersen graph given in class (where vertices are $2$-element subsets of $\{1,2,3,4,5\}$) to prove the following. 
\begin{enumerate}[(a)]
\item $G$ has no cycles of length $3$ or $4$. 
\item $G$ has no cycle of length $7$. \footnote{Hint: argue by contradiction. Observe that each vertex of $G$ has degree 3. If $C$ is a 7-cycle in $G$ is it possible that the only edges in $G$ connecting vertices in $C$ are the edges in $C$?}
\end{enumerate} 
\end{problem}

\begin{proof}

\end{proof}




\end{document}