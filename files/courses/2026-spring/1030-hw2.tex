\documentclass[11pt]{article}

\author{Math 1030}
\date{Due Friday, February 6 by 5pm} 
\title{Homework 2}

\usepackage{graphicx,xypic}
\usepackage{amsthm}
\usepackage{amsmath,amssymb}
\usepackage{amsfonts}
\usepackage{xcolor}
\usepackage[margin=1in]{geometry}
\usepackage[shortlabels]{enumitem}
\newtheorem{problem}{Problem}
\renewcommand*{\proofname}{{\color{blue}Solution}}


\setlength{\parindent}{0pt}
\setlength{\parskip}{1.25ex}


\begin{document}

\maketitle

% You are required to put your name here:
{\bf\Large Name:} 


\vspace{.3in}
Topics covered: bipartite graphs, Euler tours, vertex-degree sum formula, trees

Instructions {\bf (read these carefully!)}: 
\begin{itemize}
\item This assignment must be submitted on Gradescope by the due date. Gradescope Entry Code: 6X8XYR. 
\item If you collaborate with other students (this is encouraged),  list your collaborators above. 
\item Homework is graded anonymously, so avoid putting your name on each page of the assignment.
\item If you are stuck, please ask for help (from me, Rafi, or a classmate). Use Ed discussions!  
\item You may freely use any fact proved in class (mention it in the appropriate spot). You \emph{may not} freely use facts from the book or other sources.
\item As a general rule, try to restrict your solution to each problem to a single page. Often solutions can be short, while still being complete. If your solution is very long, think more about how you might express it more concisely.
\end{itemize}


\pagebreak 



\begin{problem}
The complete bipartite graph $K_{n,m}$ is the graph with $n+m$ vertices $v_1,\ldots,v_n$ and $u_1,\ldots,u_m$ and edges $\{v_i,u_j\}$ for each $1\le i\le n$ and $1\le j\le m$. Determine the values $n,m$ so that $K_{n,m}$ is Eulerian.
\end{problem}

\begin{proof}

\end{proof} 

\pagebreak 

\begin{problem}
Prove or disprove: 
\begin{enumerate}[(a)]
\item Every Eulerian bipartite graph has an even number of edges. 
\item Every Eulerian graph with an even number of vertices has an even number of edges. 
\end{enumerate} 
\end{problem}

\begin{proof}

\end{proof}

\pagebreak 

\begin{problem}\ \footnote{Please wait until after the lecture on Monday (2/2) to solve this.}
\begin{enumerate}[(a)]
\item Classify trees with exactly two vertices of degree $1$. 
\item What can you say about the shape of trees with either $3$ or $4$ vertices of degree $1$? (Give a qualitative statement -- you do not need to provide a formal argument.) 
\end{enumerate} 
\end{problem}

\begin{proof}

\end{proof}

\pagebreak 

\begin{problem}
Determine the number of graphs with $7$-vertices, each of degree $4$ (up to isomorphism). \footnote{Hint: consider the complement. Relate this problem to the previous one. Your solution should be short. } 
\end{problem}

\begin{proof}

\end{proof}

\pagebreak 

\begin{problem}\
\begin{enumerate}[(a)]
\item Prove that removing opposite corner squares from an $8\times 8$ checkerboard leaves a sub-board that cannot be partitioned into $1\times 2$ and $2\times 1$ rectangles. \footnote{Hint: your solution should be short.}
\item Translate your solution to the language of bipartite graphs. 
\end{enumerate} 
\end{problem}

\begin{proof}

\end{proof}

\pagebreak 

\begin{problem}
Suppose there are two mountain trails, each starting at sea level and ending at 1,000 feet. Suppose hikers $A$,$B$ start hiking these two different trails at the same time. The Mountain Climber Problem asks if it is possible for $A$ and $B$ to hike to the top of their individual trails in a way so that they have the same elevation at every time.\footnote{It is important to note that the hikers are allowed to backtrack.} We model the trails by functions $f,g:[0,1]\rightarrow[0,1]$ with $f(0)=g(0)=0$ and $f(1)=g(1)=1$. In this problem you solve the Mountain Climber Problem in the case when $f$ and $g$ are piecewise linear continuous functions.\footnote{A function $f:[0,1]\rightarrow\mathbb R$ is piecewise linear if it's possible to express $[0,1]$ as a union of finitely many intervals, so that $f$ is linear ($x\mapsto ax+b$) on each.}
\begin{enumerate}[(a)]
\item Consider \[Z=\{(x,y)\in[0,1]\times[0,1]: f(x)=g(y)\}\]
Assuming $f,g$ are piecewise linear, determine the local picture near $(x,y)$ in $Z$, considering cases based on the local pictures of $f$ and $g$ near $x$ and $y$, respectively.
\item Observe that $Z$ can be given the structure of a graph $G$. Consider vertex degrees and deduce that there is a path in $G$ from $(0,0)$ to $(1,1)$.
\end{enumerate} 
\end{problem}

\begin{proof}

\end{proof}



\end{document}