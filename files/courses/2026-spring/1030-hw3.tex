\documentclass[11pt]{article}

\author{Math 1030}
\date{Due Friday, February 13 by 5pm} 
\title{Homework 3}

\usepackage{graphicx,xypic}
\usepackage{amsthm}
\usepackage{amsmath,amssymb}
\usepackage{amsfonts}
\usepackage{xcolor}
\usepackage[margin=1in]{geometry}
\usepackage[shortlabels]{enumitem}
\newtheorem{problem}{Problem}
\renewcommand*{\proofname}{{\color{blue}Solution}}


\setlength{\parindent}{0pt}
\setlength{\parskip}{1.25ex}


\begin{document}

\maketitle

% DON'T LEAVE THIS BLANK! 
{\bf\Large Your Name:} 

% DON'T FORGET YOUR COLLABORATORS! 
Collaborator names: 

\vspace{.3in}
Topics covered: trees, Pr\"ufer codes, spanning trees, counting graphs, Bridg-it

Instructions {\bf (read these carefully!)}: 
\begin{itemize}
\item This assignment must be submitted on Gradescope by the due date. 
\item If you collaborate with other students (this is encouraged),  list your collaborators above. 
\item Homework is graded anonymously, so avoid putting your name on each page of the assignment.
\item If you are stuck, please ask for help (from me, Rafi, or a classmate). Use Ed discussions!  
\item You may freely use any fact proved in class (mention it in the appropriate spot). You \emph{may not} freely use facts from the book or other sources.
\item As a general rule, try to restrict your solution to each problem to a single page. Often solutions can be short, while still being complete. If your solution is very long, think more about how you might express it more concisely.
\end{itemize}


\pagebreak 

\begin{problem}
Find all possible graphs with the given degree sequence or prove that none exists. In either case, show your work. 
\begin{enumerate}[(a)]
\item $(3,3,2,2,2)$
\item $(5,5,4,4,2,2)$
\end{enumerate} 
\end{problem}

\pagebreak 

\begin{problem}
Determine which trees have Pr\"ufer codes that 
\begin{enumerate}[(a)]
\item contain only one value;
\item contain exactly two values;
\item have distinct values. 
\end{enumerate} 
Be sure to explain your answer. 
\end{problem}

\begin{proof}

\end{proof}


\pagebreak 

\begin{problem}
Use Prufer codes and Cayley's theorem to prove that the graph obtained from $K_n$ by deleting an edge has $(n-2)n^{n-3}$ spanning trees.\footnote{Remark/hint: One way to count how many fingers are on your right hand is to first count how many fingers you have in total and subtract the number of fingers on your left hand...}
\end{problem}

\begin{proof}

\end{proof}

\pagebreak 

\begin{problem}
Prove that the number of labeled $n$-vertex graphs where every vertex has even degree is $2^{{n-1\choose 2}}$. \footnote{Hint: establish a bijection to the set of all graphs with labeled $(n-1)$-vertex graphs.}
\end{problem}

\begin{proof}

\end{proof}

\pagebreak 

\begin{problem}
Consider the alternative version of Bridg-it where the player that connects their end-lines \emph{loses}. Use spanning trees to show that Player $2$ has a winning strategy in this game. 
\end{problem}

\begin{proof}

\end{proof}

\pagebreak 

\begin{problem}
Prove that if $T_1,\ldots,T_k$ are pairwise-intersecting subtrees of a tree $T$, then some vertex of $T$ belongs to each of $T_1,\ldots,T_k$. \footnote{Remark: This is a graph-theoretic analog of Helly's theorem.} \footnote{Hint: use induction on $k$.}
\end{problem}

\begin{proof}

\end{proof}


\end{document}