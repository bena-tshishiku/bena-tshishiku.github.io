\documentclass[11pt]{article}

\author{Math 2420}
\date{Due Friday, February 28 by 5pm} 
\title{Homework 3}

\usepackage{graphicx,xypic}
\usepackage{amsthm}
\usepackage{amsmath,amssymb}
\usepackage{amsfonts}
\usepackage{xcolor}
\usepackage[margin=1in]{geometry}
\usepackage[shortlabels]{enumitem}
\newtheorem{problem}{Problem}
\renewcommand*{\proofname}{{\color{blue}Solution}}


\setlength{\parindent}{0pt}
\setlength{\parskip}{1.25ex}


\begin{document}

\maketitle

% You are required to put your name here (this let's me know that you are submitting your own work):
{\bf\Large Your Name:} 

% You are required to put your collaborator names here:
Collaborator names: 


\vspace{.3in}
Topics covered: Hatcher 3.2 (cohomology, K\"unneth, cup products)

Instructions: 
\begin{itemize}
\item This assignment must be submitted on Gradescope by the due date. 
\item If you collaborate with other students (which is encouraged!), please list your collaborators above. 
\item Homework is graded anonymously, so please avoid putting your name on every page of the assignment.
\item If you are stuck, please ask for help (from me or a classmate). Use Campuswire!  
\item You may freely use any fact proved in class. You should provide proof for facts that you use that were not proved in class. Usually you should be able to solve the problems without outside knowledge (sometimes I overlook some fact you may need, so please do ask if it seems like you need to use a significant result from another subject). 
\item Please try to make your solutions clear and concise. When grading, I will look to see that you have the correct idea, and also that you have addressed all of the main issues (for example, if you define a map on equivalence classes, you should say why it is well defined). That being said, I encourage you to try to restrict your solution to each problem to a single page. Often solutions can be even shorter than that. 
\end{itemize}
\pagebreak 


\begin{problem}
Derive a universal coefficients theorem for homology with coefficients from the K\"unneth theorem, and use this to compute $\mathbb H_*(\mathbb RP^n;\mathbb Z/2\mathbb Z)$. Check that $\mathbb RP^n$ satisfies Poincar\'e duality $H^k\cong H_{n-k}$ with $\mathbb Z/2\mathbb Z$ coefficients. 
\end{problem}

\begin{proof}

\end{proof}

\pagebreak 

\begin{problem}
Compute the cohomology ring of the Klein bottle using simplicial homology with the (simplicial) method we used in class for the torus. 
\end{problem}

\begin{proof}

\end{proof}

\pagebreak 

\begin{problem}
Use homology or cohomology to prove that $\mathbb RP^3$ and $\mathbb RP^2\vee S^3$ are not homotopy equivalent.  
\end{problem}

\begin{proof}

\end{proof}

\pagebreak 

\begin{problem}
Prove:
\begin{enumerate}
\item[(a)] The Euler characteristic of a finite cell complex $X$ is equal to $\sum (-1)^kn_k$, where $n_k$ is the number of $k$ cells. (Recall the definition of the Euler characteristic from HW $2$.)
\item[(b)] If $Y\to X$ is a degree-$d$ covering map of cell complexes, then $\chi(Y)=d\chi(X)$. 
\end{enumerate} 
\end{problem}

\begin{proof}

\end{proof}

\pagebreak 

\begin{problem}
\begin{enumerate}[(a)]
\item Prove that every oriented surface $S$ is the boundary $S=\partial M$ of some compact $3$-manifold $M$.
\item Prove that $\mathbb R P^2$ is not the boundary of any $3$-manifold. \footnote{Hint: Proceed by contradiction. Note that a 3-manifold with boundary can be doubled to get a closed 3-manifold.} \footnote{Hint: compute the Euler characteristic of the double in two ways to reach a contradiction. It may help to use the previous exercise.}
\end{enumerate} 
\end{problem}

\begin{proof}

\end{proof}




\end{document}