\documentclass[11pt]{article}

\author{Math 2420}
\date{Due Friday, April 11 by 5pm} 
\title{Homework 5}

\usepackage{graphicx,xypic}
\usepackage{amsthm}
\usepackage{amsmath,amssymb}
\usepackage{amsfonts}
\usepackage{xcolor}
\usepackage[margin=1in]{geometry}
\usepackage[shortlabels]{enumitem}
\newtheorem{problem}{Problem}
\renewcommand*{\proofname}{{\color{blue}Solution}}


\setlength{\parindent}{0pt}
\setlength{\parskip}{1.25ex}


\begin{document}

\maketitle

% You are required to put your name here (this let's me know that you are submitting your own work):
{\bf\Large Your Name:} 

% You are required to put your collaborator names here:
Collaborator names: 


\vspace{.3in}
Topics covered: homotopy groups, LES in homotopy, fibrations

Instructions: 
\begin{itemize}
\item This assignment must be submitted on Gradescope by the due date. 
\item If you collaborate with other students (which is encouraged!), please list your collaborators above. 
\item Homework is graded anonymously, so please avoid putting your name on every page of the assignment.
\item If you are stuck, please ask for help (from me or a classmate). Use Campuswire!  
\item You may freely use any fact proved in class (mention it in the appropriate spot). You should provide proof for facts that you use that were not proved in class. Usually you should be able to solve the problems without outside knowledge (sometimes I overlook some fact you may need, so please do ask if it seems like you need to use a significant result from another subject). 
\item Please try to make your solutions clear and concise. When grading, I will look to see that you have the correct idea, and also that you have addressed all of the main issues (for example, if you define a map on equivalence classes, you should say why it is well defined). That being said, I encourage you to try to restrict your solution to each problem to a single page. Often solutions can be even shorter than that. 
\end{itemize}
\pagebreak 



\pagebreak 


\begin{problem}
Show $\mathbb R^n$ is not a union of finitely many $k$-dimensional planes when $k<n$. 
\footnote{Hint: there are many ways to do this. One option is to argue by induction on the number of planes.} \footnote{Remark: recall that this fact was used in the proof that $\pi_k(S^n)=0$ for $k<n$. } \footnote{Remark/warning: this result is false for vector spaces over finite fields.}
\end{problem}

\begin{proof}

\end{proof}


\pagebreak 



\begin{problem}
Fix an embedding of $A=S^1\vee S^1$ in $X=S^2$. Compute $\pi_2(X,A)$, and describe (explicitly) a generating set for this group. 
\end{problem}

\begin{proof}

\end{proof}

\pagebreak 


\begin{problem}
Say/compute as much as you can about the homotopy groups of $\mathbb CP^n$.  \footnote{Hint: Define a fibration over $\mathbb CP^n$ that generalizes the Hopf fibration. } 
\end{problem}


\begin{proof}

\end{proof}

\pagebreak 


\begin{problem}
Let $(B,b_0)$ be any based space. Let $PB=(B,b_0)^{(I,0)}$ denote the path space. Show that the map $p:PB\rightarrow B$ given by evaluation $p(f)=f(1)$ is a fibration. Do this by solving the lifting problem explicitly.\footnote{Hint: concatenate.} \footnote{There is a general (non-explicit) argument given in Bredon using the homotopy extension property, but I'd like you to give a direct argument.}
\end{problem}

\begin{proof}

\end{proof}

\pagebreak 


\begin{problem}
Fill in the details of the proof of the $H$-group theorem. Let $Y$ be a based space such that $X\mapsto [X,Y]$ is a functor from based spaces to groups. Show that the multiplication $\mu:Y\times Y\to Y$ defined (up to homotopy) by $[\mu]=[p_1]\cdot[p_2]\in[Y\times Y,Y]$ and the inverse $\tau:Y\to Y$ defined by $[\tau]=[1_Y]^{-1}$ makes $Y$ into an $H$-group.\footnote{Hint: the proofs should be mostly painless. A good strategy is to draw commutative diagrams.}
\end{problem}

\begin{proof}

\end{proof}



\end{document}