\documentclass[11pt]{article}

\author{Math 2420}
\date{Due Friday, February 2 by 5pm} 
\title{Homework 1}

\usepackage{graphicx,xypic}
\usepackage{amsthm}
\usepackage{amsmath,amssymb}
\usepackage{amsfonts}
\usepackage{xcolor}
\usepackage[margin=1in]{geometry}
\usepackage[shortlabels]{enumitem}
\newtheorem{problem}{Problem}
\renewcommand*{\proofname}{{\color{blue}Solution}}


\setlength{\parindent}{0pt}
\setlength{\parskip}{1.25ex}


\begin{document}

\maketitle

% You are required to put your name here:
{\bf\Large Your Name:} 

% You are required to put your collaborator names here:
Collaborator names: 


\vspace{.3in}
Topics covered: homotopy groups, H-groups, mapping spaces

Instructions: 
\begin{itemize}
\item This assignment must be submitted on Gradescope by the due date. Gradescope Entry Code: GPB45Y. 
\item If you collaborate with other students (which is encouraged!), please list your collaborators above. 
\item If you are stuck, please ask for help (from me or a classmate). Use Campuswire!  
\item You may freely use any fact proved in class. Usually you should be able to solve the problems without outside knowledge. You should provide proof for facts that you use that were not proved in class. 
\item Please restrict your solution to each problem to a single page. Usually solutions can be even shorter than that. If your solution is very long, you should think more about how to express it concisely.
\end{itemize}
\pagebreak 


\pagebreak 
\begin{problem}
Show that the set $[X,Y]$ of based homotopy classes of maps does not depend on the basepoints if $X,Y$ are path-connected. \footnote{Hint: use the homotopy extension property. Applying this probably requires some additional mild assumption on $X,Y$ (I will let you think about this).} 
\end{problem}

\begin{proof}

\end{proof}

\pagebreak 


\begin{problem}
Let $p:\tilde X\to X$ be the universal cover of a path connected space $X$. Show that $p$ induces an isomorphism on homotopy groups $\pi_k$ for $k\ge2$. 
\end{problem}

\begin{proof}

\end{proof} 


\pagebreak 

\begin{problem}For a based space $X$, $\Omega X$ denotes the loop space, and $c\in\Omega X$ denotes the constant map. Show that the map $\Omega X\rightarrow\Omega X$ defined by $\gamma\mapsto \gamma*c$ is homotopic to the identity, i.e.\ there is a homotopy $I\times \Omega X\to \Omega X$. \footnote{Remark: last semester you (probably) showed that $\gamma\simeq \gamma*c$ for each fixed $\gamma$. This is weaker than what is asked for here.}  \footnote{Remark: arguing similarly for associativity and inverses shows that $\Omega X$ is an $H$-group.} 
\end{problem}

\begin{proof}

\end{proof}

\pagebreak 

\begin{problem}
Let $A,B_1,B_2$ be based spaces. 
\begin{enumerate}[(a)]
\item Prove that $(B_1\times B_2)^A\cong B_1^A\times B_2^A$ (homeomorphism). 
\item Prove that $[A,B_1\times B_2]\cong[A,B_1]\times[A,B_2]$ (bijection of sets). 
\end{enumerate} 
\end{problem}

\begin{proof}

\end{proof}



\pagebreak 

\begin{problem}
Identify $X=\mathbb R P^\infty$ with the projectivization of the space of polynomials with coefficients in $\mathbb R$, and use this to define a monoid structure $m:X\times X\to X$. Show that the map $X\ni f\mapsto m(f,f)\in X$ is homotopic to a constant.\footnote{Hint: use the fundamental group, covering spaces.} Conclude that $\mathbb R P^\infty$ is an $H$-group. 
\end{problem}

\begin{proof}

\end{proof}





\end{document}