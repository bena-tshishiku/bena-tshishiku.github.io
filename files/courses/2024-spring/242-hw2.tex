\documentclass[11pt]{article}

\author{Math 2420}
\date{Due Friday, Feb 9 by 5pm} 
\title{Homework 2}

\usepackage{graphicx,xypic}
\usepackage{amsthm}
\usepackage{amsmath,amssymb}
\usepackage{amsfonts}
\usepackage{xcolor}
\usepackage[margin=1in]{geometry}
\usepackage[shortlabels]{enumitem}
\newtheorem{problem}{Problem}
\renewcommand*{\proofname}{{\color{blue}Solution}}


\setlength{\parindent}{0pt}
\setlength{\parskip}{1.25ex}


\begin{document}

\maketitle

% You are required to put your name here:
{\bf\Large Your Name:} 

Collaborator names: 


\vspace{.3in}
Topics covered: mapping spaces, H-groups, adjunction, etc

Instructions: 
\begin{itemize}
\item This assignment must be submitted on Gradescope by the due date. Gradescope Entry Code: GPB45Y. 
\item If you collaborate with other students (which is encouraged!), please list your collaborators above. 
\item If you are stuck, please ask for help (from me or a classmate). Use Campuswire!  
\item You may freely use any fact proved in class. Usually you should be able to solve the problems without outside knowledge. You should provide proof for facts that you use that were not proved in class. 
\item Please restrict your solution to each problem to a single page. Usually solutions can be even shorter than that. If your solution is very long, you should think more about how to express it concisely.
\end{itemize}
\pagebreak 



\begin{problem}
Assume $X$ is locally compact, and let $Y,T$ be spaces. Prove that if $H:T\to Y^X$ is continuous, then the map $h:T\times X\to Y$ defined by  $h(t,x)=H(t)(x)$ is also continuous. 
\end{problem}

\begin{proof}

\end{proof} 

\pagebreak 



\begin{problem}
Finish the proof of the $H$-group theorem. Show that the multiplication $\mu$ defined by $[\mu]=[p_1]\cdot[p_2]\in[Y\times Y,Y]$ is associative up to homotopy and has inverses up to homotopy. 
\end{problem}

\begin{proof}

\end{proof}

\pagebreak 



\begin{problem}
Explain why the correct version of the homeomorphism $Y^{T\times X}\cong (Y^X)^T$ for based spaces involves the smash product $T\wedge X$ rather than the ``ordinary" product $(T,t_0)\times (X,x_0)=\big(T\times X, (t_0,x_0)\big)$. 
\end{problem}

\begin{proof}

\end{proof}

\pagebreak 




\begin{problem}
Prove that there is no multiplication on $\mathbb R^3$ that makes it into a field. \footnote{Hint: proceed by contradiction and construct a nowhere vanishing vector field on $S^2$.} \footnote{Further hint: try fixing $u\in\mathbb R^3$ and defining vector field $F(x)=ux$. This won't quite work -- how can you fix it?}
\end{problem}

\begin{proof}

\end{proof}

\pagebreak 





\begin{problem}
Recall the special orthogonal group $SO(n)$ is the group of $n\times n$ matrices that satisfy $A^tA=I$ and $\det(A)=1$.\footnote{Note: the condition $A^tA=I$ means that the columns of $A$ form an orthonormal basis for $\mathbb R^n$.}  Let $p: SO (n+1)\rightarrow S^n$ denote the map  $A\mapsto Ae_{n+1}$, where $e_{n+1}=(0,\ldots,0,1)$. Construct a section of $p$ over $S^n\setminus\{-e_{n+1}\}$, i.e.\ construct a continuous map 
\[s:S^n\setminus\{-e_{n+1}\}\to SO (n+1)\] such that $p\circ s=id$. \footnote{Remark: the significance of this example will be explained later.} \footnote{Suggestion: do the case $n=2$ first. Then generalize.}
\end{problem}

\begin{proof}

\end{proof}




\end{document}