\documentclass[11pt]{article}

\author{Math 2420}
\date{Due Friday, March 22 by 5pm} 
\title{Homework 7}

\usepackage{graphicx,xypic}
\usepackage{amsthm}
\usepackage{amsmath,amssymb}
\usepackage{amsfonts}
\usepackage{xcolor}
\usepackage[margin=1in]{geometry}
\usepackage[shortlabels]{enumitem}
\newtheorem{problem}{Problem}
\renewcommand*{\proofname}{{\color{blue}Solution}}


\setlength{\parindent}{0pt}
\setlength{\parskip}{1.25ex}


\begin{document}

\maketitle

% You are required to put your name here:
{\bf\Large Your Name:} 

Collaborator names: 


\vspace{.3in}
Topics covered: cohomology, universal coefficients

Instructions: 
\begin{itemize}
\item This assignment must be submitted on Gradescope by the due date. 
\item If you collaborate with other students (encouraged!), please list your collaborators above. 
\item If you are stuck, please ask for help (from me or a classmate). Use Campuswire!  
\item You may freely use any fact proved in class. Usually you should be able to solve the problems without outside knowledge. You should provide proof for facts that you use that were not proved in class. 
\item Please restrict your solution to each problem to a single page. Usually solutions can be even shorter than that. If your solution is very long, you should think more about how to express it concisely.
\end{itemize}
\pagebreak 





\begin{problem}
Let $M$ be an abelian group. Show that $Hom(-,M)$ preserves split short exact sequences of abelian groups, i.e.\ if $0\rightarrow A\xrightarrow{i}B\xrightarrow{p}C\rightarrow0$ is a split short exact sequence, then 
\[0\rightarrow Hom(C,M)\xrightarrow{p^*} Hom(B,M)\xrightarrow{i^*} Hom(A,M)\rightarrow0\]
is also a split short exact sequence. 
\end{problem} 

\begin{proof}

\end{proof}

\pagebreak 

\begin{problem}
Compute the cellular homology of $X=\mathbb RP^2\vee S^3$ (with the obvious cell structure). Use universal coefficients to compute $H^*(X;\mathbb Z)$ and $H^*(X;\mathbb Z/2\mathbb Z)$. \footnote{Hint: computation should look familiar.} 
\end{problem}

\begin{proof}

\end{proof}

\pagebreak 

\begin{problem}
True or false: the functors $h_k(X,A) = Hom(H_k(X,A), \mathbb Z)$ defines (co)homology theory (in the sense of Eilenberg--Steenrod axioms).
\end{problem}

\begin{proof}

\end{proof}

\pagebreak 

\begin{problem}
Let $(X,x_0)$ be a connected cell complex. Show that every homomorphism $\phi:\pi_1(X,x_0)\rightarrow\mathbb Z$ is induced by a map $f:(X, x_0)\rightarrow(S^1,s_0)$ that is unique up to homotopy that's constant on $x_0$. \footnote{Hint: It may help to think about the argument that Eilenberg--Maclane spaces are unique up to homotopy equivalence.} \footnote{Remark: the connection of this problem with cohomology will appear on the next assignment. } 
\end{problem}

\begin{proof}

\end{proof}



\end{document}