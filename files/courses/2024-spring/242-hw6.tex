\documentclass[11pt]{article}

\author{Math 2420}
\date{Due Friday, March 8 by 5pm} 
\title{Homework 6}

\usepackage{graphicx,xypic}
\usepackage{amsthm}
\usepackage{amsmath,amssymb}
\usepackage{amsfonts}
\usepackage{xcolor}
\usepackage[margin=1in]{geometry}
\usepackage[shortlabels]{enumitem}
\newtheorem{problem}{Problem}
\renewcommand*{\proofname}{{\color{blue}Solution}}


\setlength{\parindent}{0pt}
\setlength{\parskip}{1.25ex}


\begin{document}

\maketitle

% You are required to put your name here:
{\bf\Large Your Name:} 

Collaborator names: 


\vspace{.3in}
Topics covered: cellular approximation, Whitehead's theorem, fiber bundles

Instructions: 
\begin{itemize}
\item This assignment must be submitted on Gradescope by the due date. 
\item If you collaborate with other students (encouraged!), please list your collaborators above. 
\item If you are stuck, please ask for help (from me or a classmate). Use Campuswire!  
\item You may freely use any fact proved in class. Usually you should be able to solve the problems without outside knowledge. You should provide proof for facts that you use that were not proved in class. 
\item Please restrict your solution to each problem to a single page. Usually solutions can be even shorter than that. If your solution is very long, you should think more about how to express it concisely.
\end{itemize}
\pagebreak 


\pagebreak 



\begin{problem}
Show the double comb space is not contractible.\footnote{Hint: Let $x_0$ be the wedge point. It may help to start by showing that there is no deformation retract to $x_0$. Use continuity to argue that there are small neighborhoods $x_0\in V\subset U$ so that $f_t(V)\subset U$ for all $t$. Use this to reach a contradiction. The general argument should work similarly.} 
\end{problem}

\begin{proof}

\end{proof}

\pagebreak

\begin{problem}[Hatcher 4.1]
Show that a cell complex $X$ is contractible if it is the union of an increasing sequence of subcomplexes $X_1\subset X_2\subset \cdots$ such that each inclusion $X_i\hookrightarrow X_{i+1}$ is nullhomotopic. \footnote{Hint: your solution should be short.} \footnote{Remark: note that this implies that $S^\infty$ is contractible, as similarly the union of iterated suspensions $\Sigma^nX$ for any space $X$.} 
\end{problem}

\begin{proof}

\end{proof} 


\pagebreak

\begin{problem}
Show that any two $K(A,n)$ spaces are homotopy equivalent.\footnote{Hint: Construct a weak homotopy equivalence inductively. You can use cellular approximation of spaces, to make your life a little simpler.} Where does this argument break down in trying to show that $\mathbb RP^2$ and $\mathbb S^2\times\mathbb RP^\infty$ are homotopy equivalent? 
\end{problem}

\begin{proof}

\end{proof}

\pagebreak

\begin{problem}
Associated to a fiber bundle $F\to E\to S^1$, there is a homeomorphism $\phi:F\to F$ call the monodromy, which is well-defined up to isotopy. It's defined as the ``first return map" of a sequence of local trivializations around the circle.\footnote{For example, the monodromy of the $n$-fold covering map $S^1\to S^1$ is a cyclic permutation. Understand this!} For the trefoil knot complement fibering $F\to S^3\setminus K\to S^1$, compute the action of  the monodromy on $H_1(F)$.\footnote{Hint: start by fixing a basis, so that the answer is a $2\times 2$ matrix.} 
\end{problem}


\begin{proof}

\end{proof}




\end{document}