\documentclass[11pt]{article}

\author{Math 2420}
\date{Due Friday, Feb 16 by 5pm} 
\title{Homework 4}

\usepackage{graphicx,xypic}
\usepackage{amsthm}
\usepackage{amsmath,amssymb}
\usepackage{amsfonts}
\usepackage{xcolor}
\usepackage[margin=1in]{geometry}
\usepackage[shortlabels]{enumitem}
\newtheorem{problem}{Problem}
\renewcommand*{\proofname}{{\color{blue}Solution}}


\setlength{\parindent}{0pt}
\setlength{\parskip}{1.25ex}


\begin{document}

\maketitle

% You are required to put your name here:
{\bf\Large Your Name:} 

Collaborator names: 


\vspace{.3in}
Topics covered: fibrations, fiber bundles, $\pi_n(S^n)$

Instructions: 
\begin{itemize}
\item This assignment must be submitted on Gradescope by the due date. 
\item If you collaborate with other students (encouraged!), please list your collaborators above. 
\item If you are stuck, please ask for help (from me or a classmate). Use Campuswire!  
\item You may freely use any fact proved in class. Usually you should be able to solve the problems without outside knowledge. You should provide proof for facts that you use that were not proved in class. 
\item Please restrict your solution to each problem to a single page. Usually solutions can be even shorter than that. If your solution is very long, you should think more about how to express it concisely.
\end{itemize}
\pagebreak 


\pagebreak 


\begin{problem}
Use the connecting homomorphism in the long exact sequence of the Hopf fibration $p:S^3\to S^2$ to argue that $\pi_2(S^2)$ is generated by the identity map. \footnote{Hint: It may help to consider $\mathbb C\to\mathbb C P^1$ defined by $z\mapsto[z:1]$ and construct a lifting of this map to $S^3$, or even $\mathbb C^2\setminus0$.}
\end{problem}

\begin{proof}

\end{proof}

\pagebreak 

\begin{problem}
Let $p:E\to B$ be a Serre fibration and assume $B$ is connected. Fix $b,b'\in B$ with fibers $F,F'$, respectively. Define a homomorphism $\pi_k(F)\to\pi_k(F')$, and show that it's a bijection. \footnote{Hint/suggestion: revisit the similar argument from class. Be thorough and give details for parts of the argument that were skimmed over in lecture.} 
\end{problem}

\begin{proof}

\end{proof}

\pagebreak 

\begin{problem}
Show every map $f:S^n\to S^n$ is homotopic to a multiple of the identity map by the following steps. 
\begin{enumerate}[(a)]
\item Use local PL lemma to reduce to the case that there exists $q\in S^n$ with $f^{-1}(q)=\{p_1,\ldots,p_k\}$ and $f$ is an invertible linear map near each $p_i$. 
\item For $f$ as in (a), consider the composition $g\circ f$ where $g:S^n\to S^n$ collapses the complement of a small ball about $q$ to the basepoint. Use this to reduce (a) further to the case $k=1$. \footnote{Hint: Express $f$ as a concatenation.}
\item Finish the argument by showing that an invertible $n\times n$ matrix can be joined by a path of such matrices to either the identity matrix or a reflection. 
\end{enumerate} 
\end{problem}

\begin{proof}

\end{proof} 

\pagebreak 

\begin{problem}
Give a decomposition of the trefoil knot complement $S^3\setminus K$ into a union of copies of a punctured torus, index by points in the circle.\footnote{Remark: in fact there is a fiber bundle $S^3\setminus K\to S^1$, where the fibers are punctured tori. This fiber bundle structure gives a decomposition of $S^3\setminus K$ into a union of fibers. I'm not asking you to prove there's a fiber bundle, only to explain the decomposition into a union of fibers.} \footnote{Hint: it's helpful to recall that $S^3$ is a union of solid tori $S^1\times D^2$ and $D^2\times S^1$ (an ``inside" and an ``outside" solid torus), glued along a torus $S^1\times S^1$. Put the trefoil knot on the surface of the torus. The fibers can be described in terms of their intersections with the solid tori. You can do this so that each fiber intersects the outside solid torus in two disjoint (meridinal) disks. This should help to pin down how to choose the intersection with the inside solid torus.} 
\footnote{Hint: it's likely that some combination of pictures and explanation are most effective for expressing a solution.}
\end{problem}

\begin{proof}

\end{proof}



\end{document}