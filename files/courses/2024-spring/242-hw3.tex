\documentclass[11pt]{article}

\author{Math 2420}
\date{Due Friday, Feb 16 by 5pm} 
\title{Homework 3}

\usepackage{graphicx,xypic}
\usepackage{amsthm}
\usepackage{amsmath,amssymb}
\usepackage{amsfonts}
\usepackage{xcolor}
\usepackage[margin=1in]{geometry}
\usepackage[shortlabels]{enumitem}
\newtheorem{problem}{Problem}
\renewcommand*{\proofname}{{\color{blue}Solution}}


\setlength{\parindent}{0pt}
\setlength{\parskip}{1.25ex}


\begin{document}

\maketitle

% You are required to put your name here:
{\bf\Large Your Name:} 

Collaborator names: 


\vspace{.3in}
Topics covered: relative homotopy groups, LES of a pair, fibrations

Instructions: 
\begin{itemize}
\item This assignment must be submitted on Gradescope by the due date. 
\item If you collaborate with other students (encouraged!), please list your collaborators above. 
\item If you are stuck, please ask for help (from me or a classmate). Use Campuswire!  
\item You may freely use any fact proved in class. Usually you should be able to solve the problems without outside knowledge. You should provide proof for facts that you use that were not proved in class. 
\item Please restrict your solution to each problem to a single page. Usually solutions can be even shorter than that. If your solution is very long, you should think more about how to express it concisely.
\end{itemize}
\pagebreak 


\pagebreak 


\begin{problem}
Show $\mathbb R^n$ is not a union of finitely many $k$-dimensional planes when $k<n$. 
\footnote{Hint: it's possible to argue by induction on the number of planes.} \footnote{Remark: this fact was used in the proof that $\pi_k(S^n)=0$ for $k<n$. } \footnote{Remark: this result is false for vector spaces over finite fields.}
\end{problem}

\begin{proof}

\end{proof}


\pagebreak

\begin{problem}
Compute/say as much as you can about the homotopy groups of $\mathbb CP^n$.  \footnote{Hint: Define a fibration over $\mathbb CP^n$ that generalizes the Hopf fibration. } 
\end{problem}


\begin{proof}

\end{proof}

\pagebreak

\begin{problem}
Let $(B,b_0)$ be any based space. Let $PB=(B,b_0)^{(I,0)}$ denote the path space. Show that the map $p:PB\rightarrow B$ given by evaluation $p(f)=f(1)$ is a fibration. Do this by solving the lifting problem explicitly. \footnote{Hint: concatenate.}
\end{problem}

\begin{proof}

\end{proof}

\pagebreak

\begin{problem}
Fix an embedding of $A=S^1\vee S^1$ in $X=S^2$. Compute $\pi_2(X,A)$, and describe a generating set for this group. 
\end{problem}

\begin{proof}

\end{proof}

\pagebreak

\begin{problem}
We say $p:E\to B$ has local sections if for each $b\in B$, there is section of $p$ defined on an open neighborhood of $b$. Let $G$ be a topological group, let $H<G$ be a subgroup, and consider the quotient map $p:G\to G/H$. Prove that if $p$ has local sections, then $p$ is locally trivial.\footnote{Hint: observe that the action of $H$ on $G$ by right multiplication preserves each fiber of $p$.} \footnote{Remark: on HW2 you showed that $SO(n+1)\to S^n$ has local sections.} 
\end{problem}

\begin{proof}

\end{proof}



\end{document}