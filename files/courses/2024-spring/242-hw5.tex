\documentclass[11pt]{article}

\author{Math 2420}
\date{Due Friday, March 1 by 5pm} 
\title{Homework 5}

\usepackage{graphicx,xypic}
\usepackage{amsthm}
\usepackage{amsmath,amssymb}
\usepackage{amsfonts}
\usepackage{xcolor}
\usepackage[margin=1in]{geometry}
\usepackage[shortlabels]{enumitem}
\newtheorem{problem}{Problem}
\renewcommand*{\proofname}{{\color{blue}Solution}}


\setlength{\parindent}{0pt}
\setlength{\parskip}{1.25ex}


\begin{document}

\maketitle

% You are required to put your name here:
{\bf\Large Your Name:} 

Collaborator names: 


\vspace{.3in}
Topics covered: homotopy excision, cellular maps

Instructions: 
\begin{itemize}
\item This assignment must be submitted on Gradescope by the due date. 
\item If you collaborate with other students (encouraged!), please list your collaborators above. 
\item If you are stuck, please ask for help (from me or a classmate). Use Campuswire!  
\item You may freely use any fact proved in class. Usually you should be able to solve the problems without outside knowledge. You should provide proof for facts that you use that were not proved in class. 
\item Please restrict your solution to each problem to a single page. Usually solutions can be even shorter than that. If your solution is very long, you should think more about how to express it concisely.
\end{itemize}
\pagebreak 


\pagebreak 


\begin{problem}
Fix a pair $(X,A)$. Assume that $(X,A)$ is $n$-connected and $A$ is $m$-connected. Prove that $\pi_k(X,A)\cong\pi_k(X/A)$ for $k\le n+m$. \footnote{Hint: replace $X/A$ with $X\cup CA$ (where $CA$ is the cone), and apply the LES of a pair and excision. Your solution should be pretty short.}
\end{problem}

\begin{proof}

\end{proof}


\pagebreak 

\begin{problem}
For $A=S^1\vee S^1$ embedded in $S^2$, compute $\pi_2(X/A)$. Compare with $\pi_2(X,A)$. \footnote{Remark: this example shows the failure of excision for homotopy groups.} 
\end{problem}

\begin{proof}

\end{proof} 


\pagebreak 

\begin{problem}
Let $X,Y$ be cell complexes and assume they are homotopy equivalent. Show that the $n$-skeleta $X^{(n)}$ and $Y^{(n)}$ are homotopy equivalent if $X,Y$ don't have $(n+1)$-dimensional cells. Give an example to show this assumption on $(n+1)$-cells is necessary. 
\end{problem}

\begin{proof}

\end{proof} 

\pagebreak 

\begin{problem}
Generally, for a path $\gamma:I\to X$, there is a change of basepoint homomorphism $\pi_k(X,x_0)\to \pi_k(X,x_1)$, where $x_i=\gamma(i)$ for $i=0,1$. See Hatcher pg.\ 341. Taking $\gamma$ to be a loop based at $x_0$ defines an action of $\pi_1(X,x_0)$ on $\pi_k(X,x_0)$ for $k\ge1$. 
\begin{enumerate}
\item[(a)] Let $i:S^2\hookrightarrow S^2\vee S^1$ be the inclusion, viewed as an element of $\pi_2(S^2\vee S^1)$. Describe the orbit of $i$ under the action of $\pi_1(S^2\vee S^1)\cong\mathbb Z$. \footnote{Remark: Here ``describe" means in explicit geometric terms, e.g.\ in terms of representative maps.} \footnote{Remark: It is also informative to think about how the action looks on $\pi_2$ of the universal cover.}
\item[(b)] Compute the action of $\pi_1(\mathbb RP^n)$ on $\pi_n(\mathbb RP^n)=\mathbb Z$. 
\footnote{Hint: the answer depends on whether $n$ is even or odd.}
\end{enumerate} 
\end{problem}

\begin{proof}

\end{proof}


\end{document}