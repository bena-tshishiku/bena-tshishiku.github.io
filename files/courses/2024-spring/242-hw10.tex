\documentclass[11pt]{article}

\author{Math 2420}
\date{Due Friday, April 19 by 5pm} 
\title{Homework 10}

\usepackage{graphicx,xypic}
\usepackage{amsthm}
\usepackage{amsmath,amssymb}
\usepackage{amsfonts}
\usepackage{xcolor}
\usepackage[margin=1in]{geometry}
\usepackage[shortlabels]{enumitem}
\newtheorem{problem}{Problem}
\renewcommand*{\proofname}{{\color{blue}Solution}}


\newcommand{\Ext}{Ext}
\newcommand{\Hom}{Hom}
\newcommand{\Z}{\mathbb Z}
\setlength{\parindent}{0pt}
\setlength{\parskip}{1.25ex}


\begin{document}

\maketitle

% You are required to put your name here:
{\bf\Large Your Name:} 

Collaborator names: 


\vspace{.3in}
Topics covered: cohomology, cup products, Euler characteristic

Instructions: 
\begin{itemize}
\item This assignment must be submitted on Gradescope by the due date. 
\item If you collaborate with other students (encouraged!), please list your collaborators above. 
\item If you are stuck, please ask for help (from me or a classmate). Use Campuswire!  
\item You may freely use any fact proved in class. Usually you should be able to solve the problems without outside knowledge. You should provide proof for facts that you use that were not proved in class. 
\item Please restrict your solution to each problem to a single page. Usually solutions can be even shorter than that. If your solution is very long, you should think more about how to express it more concisely.
\end{itemize}
\pagebreak 


\begin{problem}
For a chain complex $C$ and abelian group $M$, the homology with coefficients in $M$, denoted $H_*(C;M)$, is defined as the homology of the chain complex $C\otimes M$. Derive a universal coefficients theorem for homology with coefficients from the K\"unneth theorem, and use this to compute $\mathbb H_*(\mathbb RP^n;\mathbb Z/2\mathbb Z)$. Compare with $H^*(\mathbb RP^2;\mathbb Z/2Z)$.
\end{problem}

\begin{proof}

\end{proof}

\pagebreak 

\begin{problem}
Compute the cup product structure on $\mathbb RP^2$ using simplicial homology.
\end{problem}

\begin{proof}

\end{proof}

\pagebreak 

\begin{problem}
Use homology or cohomology to prove that $\mathbb RP^3$ and $\mathbb RP^2\vee S^3$ are not homotopy equivalent.  
\end{problem}

\begin{proof}

\end{proof}

\pagebreak 

\begin{problem}
Prove:
\begin{enumerate}
\item[(a)] The Euler characteristic of a finite cell complex $X$ is equal to $\sum (-1)^kn_k$, where $n_k$ is the number of $k$ cells.
\item[(b)] If $Y\to X$ is a degree-$d$ covering map of cell complexes, then $\chi(Y)=d\chi(X)$. 
\end{enumerate} 
\end{problem}

\begin{proof}

\end{proof}

\pagebreak 

\begin{problem}
\begin{enumerate}[(a)]
\item Prove that every oriented surface $S$ is the boundary $S=\partial M$ of some compact $3$-manifold $M$.
\item Prove that $\mathbb R P^2$ is not the boundary of any $3$-manifold. \footnote{Hint: Proceed by contradiction. Note that a 3-manifold with boundary can be doubled to get a closed 3-manifold.} \footnote{Hint: compute the Euler characteristic of the double in two ways to reach a contradiction. You may find another problem on this assignment helpful.}
\end{enumerate} 
\end{problem}

\begin{proof}

\end{proof}




\end{document}